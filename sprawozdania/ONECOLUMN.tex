\documentclass[12pt,twoside,a4paper,twocolumn]{article} 
\usepackage[english,polish]{babel}
\usepackage{polski}
\usepackage{xcolor}
\usepackage{graphicx}
\begin{document}

pogrubienie \textbf{} zgubne frytki \textbf{zgubne frytki}\\
normalna grubo"s"c \textmd{} zgubne frytki \textmd{zgubne frytki}\\
kursywa \textit{} zgubne frytki \textit{zgubne frytki}\\
pochylenie \textsl{} zgubne frytki \textsl{zgubne frytki}\\
podkre"slenie  \underline{} zgubne frytki \underline{zgubne frytki}\\
pismo proste \textup{} zgubne frytki \textup{zgubne frytki}\\
kapitaliki \textsc{} ZGUBNE FRYTKI \textsc{zgubne frytki}\\
czcionka o sta"lej szeroko"sci \texttt{} zgubne frytki \texttt{zgubne frytki}\\
czcionka bezszeryfowa \textsf{} zgubne frytki \textsf{zgubne frytki}\\
czcionka typu antykwa \textrm{} zgubne frytki \textrm{zgubne frytki}\\

tiny Miłorz"eby \tiny{Miłorz"eby}\\
scriptsize Miłorz"eby \scriptsize{Miłorz"eby}\\
footnotesize Miłorz"eby \footnotesize{Miłorz"eby}\\
small Miłorz"eby\small{Miłorz"eby}\\
normalsize Miłorz"eby \normalsize{Miłorz"eby}\\
large Miłorz"eby \large{Miłorz"eby}\\
Large Miłorz"eby \Large{Miłorz"eby}\\
LARGE Miłorz"eby \LARGE{Miłorz"eby}\\
huge Miłorz"eby \huge{Miłorz"eby}\\
Huge Miłorz"eby \Huge{Miłorz"eby}\\


\begin{tabular}{lr|c}
towar & waga netto & cena\\\hline
szyneczka wieprzowa & 1kg & 22zł\\
pasztet drobiowy & 175g & 1zł\\
patyczki do uszu & 20kg & 130zł
\end{tabular}
\\
\\
\\

\textcolor[RGB]{255,214,76}{kolorowe kredki}\\
\textcolor{brown}{kolorowe kredki}\\
\textcolor{darkgray}{kolorowe kredki}\\
\textcolor{orange}{kolorowe kredki}\\


\newpage
Jakieś zdanie.
$$ a^{2} $$ 
Jakieś zdanie.
\[ a^{2} \]
Jakieś zdanie.
\begin{displaymath}
c^{2}=a^{2}+b^{2}
\end{displaymath}
Jakieś zdanie.
\begin{equation}
\epsilon > 0 \label{eq:eps}
\end{equation}
Ze wzoru (\ref{eq:eps}) otrzymujemy \ldots\\
\\
\\

$\lim_{n \to \infty}
\sum_{k=1}^n \frac{1}{k^2}
= \frac{\pi^2}{6}$
\\ \\
$$
\lim_{n \to \infty}
\sum_{k=1}^n \frac{1}{k^2}
= \frac{\pi^2}{6}
$$
\\
\\
$$
y = \left\{ \begin{array}{ll}
a & \textrm{gdy $d>c$}\\
b+x & \textrm{gdy $d=c$}\\
l & \textrm{gdy $ d < c $}
\end{array} \right.
$$

\end{document}